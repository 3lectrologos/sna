\documentclass[10pt]{article} % For LaTeX2e
\usepackage{nips12submit_e,times}
%\documentstyle[nips12submit_09,times,art10]{article} % For LaTeX 2.09

\usepackage[utf8]{inputenc}
\usepackage[T1]{fontenc}
\usepackage{textcomp}
\usepackage{lmodern}

\usepackage{amsmath,amssymb}
\usepackage{booktabs}
\usepackage{graphicx}
\usepackage{color}

\usepackage{url}
\usepackage{hyperref}

\newcommand{\todo}[1]{\noindent\texttt{\color[rgb]{0.5,0.1,0.1} TODO: #1}}

\title{Introduction to Social Network Analysis:\\Final Project}

\author{
Alkis Gkotovos\\
Department of Computer Science, ETH Zurich\\
\texttt{alkisg@student.ethz.ch}
}

% The \author macro works with any number of authors. There are two commands
% used to separate the names and addresses of multiple authors: \And and \AND.
%
% Using \And between authors leaves it to \LaTeX{} to determine where to break
% the lines. Using \AND forces a linebreak at that point. So, if \LaTeX{}
% puts 3 of 4 authors names on the first line, and the last on the second
% line, try using \AND instead of \And before the third author name.

\newcommand{\fix}{\marginpar{FIX}}
\newcommand{\new}{\marginpar{NEW}}

\nipsfinalcopy % Uncomment for camera-ready version

\begin{document}


\maketitle

\begin{abstract}
In this report I present a model, similar to the common SIR and SIS models,
which can be used to simulate the spread of an infectious disease in
social networks. I also present and discuss simulation results on several
kinds of generated networks for different choices of model and graph parameters.
\end{abstract}

\section{Introduction}
The spread of diseases in human networks is an interesting research topic, both
for historical reasons (e.g. analysis of the ``Black Death'' pandemic in the
14th century~\cite{blackdeath}), as well as for dealing with contemporary or
future epidemics (e.g. AIDS spread in Africa~\cite{aids}). Furthermore, the
concept of network diffusion can be generalized to other types of phenomena,
like the spread of computer viruses in e-mail networks~\cite{email}, or even
the spread of behavioral phenomena, also referred to as
\emph{social contagion}~\cite{contagion}.

To fully analyze disease spread in networks, one needs to know in detail the
characteristics of the disease itself (transimission, active period, etc.),
as well as a fine-grained network of interactions. Moreover, the network
information required may depend on the disease attributes (e.g. for an airborne virus
it may be important to include interactions in public transport). Since this
kind of information is usually not available, there have been proposed a
number of simple models that can be used as a first step for analyzing disease
spread. Two of them commonly found in the literature are the SIR and SIS
models~\cite{easley, newman}.

Both models assume discrete time steps and represent each node of the network
as a state machine that can be in exactly one state at each time step. In the
SIR model healthy nodes that can be infected are in the \emph{susceptible} ($S$)
state, infected nodes are in the \emph{infected} (I) state, and nodes that have
gone through the infection and cannot be infected again are in the \emph{removed}
($R$) state. At every time step, an $I$ node may independently infect each of
its $S$ neighbors with probability $p_I$ and an $I$ node stays infected for
$t_I$ steps and, afterwards, transitions into the $R$ state.
The SIS model is very similar and only differs in the fact that nodes return
to the $S$ state after the infection is over, instead of transitioning to the
$R$ state.

\section{Model description}


\bibliographystyle{plain}
\bibliography{report}

\end{document}
